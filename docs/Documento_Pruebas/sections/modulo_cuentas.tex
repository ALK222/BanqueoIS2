\section{Módulo cuentas}
\subsection{Crear cuenta}
\begin{table}[H]
    \centering
    \begin{tabularx}{\textwidth}{|>{\bfseries}X|X|}
        \hline
        Nombre de la función                               & altaCuenta                                                                                \\
        \hline
        Precondición                                       & 1.- El gestor inició sesión. \newline 2.- La cuenta no debe existir previamente a su alta \\
        \hline
        Postcondición                                      & Los datos de la cuenta estarán en la base de datos.                                       \\
        \hline
        Flujo principal                                    &
        1.- La función cumple con la precondición.\newline
        2.- Los datos de entrada son los datos de la nueva cuenta.\newline
        3.- La salida muestra una pantalla diciendo que la cuenta se ha creado satisfactoriamente.\newline
        4.- La función cumple con la postcondición.\newline
        \\
        \hline
        Flujo alternativo 1 \newline (La cuenta ya existe) &
        1.- La función cumple no con la precondición.\newline
        2.- Los datos de entrada son los datos de la nuevo cuenta.\newline
        3.- La salida muestra una pantalla diciendo que la cuenta ya existe.\newline
        4.- La función cumple con la postcondición.\newline                                                                                            \\
        \hline
    \end{tabularx}
\end{table}

\subsection{Eliminar cuenta}
\begin{table}[H]
    \centering
    \begin{tabularx}{\textwidth}{|>{\bfseries}X|X|}
        \hline
        Nombre de la función                            & bajaCuenta                                                                              \\
        \hline
        Precondición                                    & 1.- El gestor inició sesión. \newline 2.- La cuenta debe existir previamente a su alta. \\
        \hline
        Postcondición                                   & Los datos de la cuenta no estarán en la base de datos.                                  \\
        \hline
        Flujo principal                                 &
        1.- La función cumple con la precondición.\newline
        2.- Los datos de entrada son los datos de la cuenta a dar de baja.\newline
        3.- La salida muestra una pantalla diciendo que la cuenta se ha eliminado satisfactoriamente.\newline
        4.- La función cumple con la postcondición.\newline
        \\
        \hline
        Flujo alternativo 1 \newline (Cuenta no existe) &
        1.- La función cumple no con la precondición.\newline
        2.- Los datos de entrada son los datos de la cuenta a dar de baja.\newline
        3.- La salida muestra una pantalla diciendo que la cuenta no existe.\newline
        4.- La función cumple con la postcondición.\newline                                                                                       \\
        \hline
    \end{tabularx}
\end{table}

\subsection{Consultar la lista de cuentas}
\begin{table}[H]
    \centering
    \begin{tabularx}{\textwidth}{|>{\bfseries}X|X|}
        \hline
        Nombre de la función                                              & consultarCuentas                                                                          \\
        \hline
        Precondición                                                      & 1.- El cuenta inició sesión. \newline 2.- Los parámetros de búsqueda deben ser correctos. \\
        \hline
        Postcondición                                                     & Se muestran las cuentas que pasan por el filtro.                                          \\
        \hline
        Flujo principal                                                   &
        1.- La función cumple con la precondición.\newline
        2.- Los datos de entrada son dos string: uno con la DNI del titular y el otro su nombre.\newline
        3.- La salida muestra una lista con las cuentas que pasan el filtro.\newline
        4.- La función cumple la postcondición.\newline
        \\
        \hline
        Flujo alternativo 1 \newline (Parámetros de búsqueda incorrectos) &
        1.- La función cumple no con la precondición.\newline
        2.- Los datos de entrada son un string DNI y el nombre.\newline
        3.- La salida muestra una pantalla de error.\newline
        4.- La función no cumple la postcondición.\newline                                                                                                            \\
        \hline
        Flujo alternativo 2 \newline (Ningún cuenta pasa el filtro)       &
        1.- La función cumple no con la precondición.\newline
        2.- Los datos de entrada son dos string: uno con la DNI del titular y el otro su nombre.\newline
        3.- La salida muestra una lista vacía, ya que ninguna cuenta pasa el filtro.\newline
        4.- La función no cumple la postcondición.\newline                                                                                                            \\
        \hline
    \end{tabularx}
\end{table}

\subsection{Buscar cuenta}

\begin{table}[H]
    \centering
    \begin{tabularx}{\textwidth}{|>{\bfseries}X|X|}
        \hline
        Nombre de la función                                                                        & buscarCuenta                                                                            \\
        \hline
        Precondición                                                                                & 1.- El gestor inició sesión. \newline 2.- La cuenta debe existir previamente a su alta. \\
        \hline
        Postcondición                                                                               & Se mostrarán los datos de dicha cuenta.                                                 \\
        \hline
        Flujo principal                                                                             &
        1.- La función cumple con la precondición.\newline
        2.- El dato de entrada es el número de referencia de la cuenta a buscar.\newline
        3.- La salida muestra los datos de la cuenta.\newline
        4.- La función cumple con la postcondición.\newline
        \\
        \hline
        Flujo alternativo 1 \newline (El número de referencia no se corresponde con ninguna cuenta) &
        1.- La función cumple no con la precondición.\newline
        2.- El dato de entrada es el número de referencia de la cuenta a buscar.\newline
        3.- La salida muestra una pantalla diciendo que la cuenta no existe.\newline
        4.- La función no cumple con la postcondición.\newline                                                                                                                                \\
        \hline
    \end{tabularx}
\end{table}

\subsection{Modificar cuentas}
\begin{table}[H]
    \centering
    \begin{tabularx}{\textwidth}{|>{\bfseries}X|X|}
        \hline
        Nombre de la función                                     & modificarCuenta                                                                         \\
        \hline
        Precondición                                             & 1.- El gestor inició sesión. \newline 2.- La cuenta debe existir previamente a su alta. \\
        \hline
        Postcondición                                            & Se modifican los datos de la cuenta en la base de datos.                                \\
        \hline
        Flujo principal                                          &
        1.- La función cumple con la precondición.\newline
        2.- El dato de entrada es la cuenta modificada.\newline
        3.- La salida muestra una ventana que indica que la operación se ha realizado correctamente.\newline
        4.- La función cumple con la postcondición.\newline
        \\
        \hline
        Flujo alternativo 1 \newline (No se guardan los cambios) &
        1.- La función cumple no con la precondición.\newline
        2.- El dato de entrada es la cuenta modificada.\newline
        3.- La salida muestra una pantalla de error.\newline
        4.- La función no cumple con la postcondición.\newline                                                                                             \\
        \hline
    \end{tabularx}
\end{table}

\subsection{Consultar movimientos}
\begin{table}[H]
    \centering
    \begin{tabularx}{\textwidth}{|>{\bfseries}X|X|}
        \hline
        Nombre de la función                                            & getMovimientos                                                                        \\
        \hline
        Precondición                                                    & 1.- El cuenta inició sesión. \newline 2.- La cuenta debe existir en la base de datos. \\
        \hline
        Postcondición                                                   & Se muestran los movimientos que pasan por el filtro.                                  \\
        \hline
        Flujo principal                                                 &
        1.- La función cumple con la precondición.\newline
        2.- Los datos de entrada son los meses a consultar y la cuenta.\newline
        3.- La salida muestra una lista con las cuentas que pasan el filtro.\newline
        4.- La función cumple la postcondición.\newline
        \\
        \hline
        Flujo alternativo 1 \newline (La cuenta es errónea)             &
        1.- La función cumple no con la precondición.\newline
        2.- Los datos de entrada son los meses a consultar y la cuenta.\newline
        3.- La salida muestra una pantalla de error.\newline
        4.- La función no cumple la postcondición.\newline                                                                                                      \\
        \hline
        Flujo alternativo 2 \newline (Ningún movimiento pasa el filtro) &
        1.- La función cumple no con la precondición.\newline
        2.- Los datos de entrada son los meses a consultar y la cuenta.\newline
        3.- La salida muestra una lista vacía, ya que ningún movimiento pasa el filtro.\newline
        4.- La función no cumple la postcondición.\newline                                                                                                      \\
        \hline
    \end{tabularx}
\end{table}
