\documentclass[12pt]{article}
\usepackage[a4paper, margin=1in, headheight=14pt]{geometry}
\usepackage[utf8]{inputenc}
\usepackage[spanish]{babel}
\usepackage{tabularx}
\usepackage{graphicx}
\usepackage{float}
\usepackage{setspace}
\usepackage{anyfontsize}
\usepackage[toc,page]{appendix} % Remove toc,page to render wo appendix
\usepackage{hyperref}
\usepackage[table]{xcolor}
\usepackage{fancyhdr}
\usepackage{nameref}
\usepackage{datetime}
\usepackage{tikz}
\usepackage[11pt]{moresize}
\usepackage{enumerate}
\usepackage{etoolbox}


\usetikzlibrary{calc}
\newcommand\HRule{\rule{\textwidth}{1pt}}


\pagestyle{fancy}
\fancyhf{}
\fancyhead[LE,RO]{Grupo 4}
\fancyhead[RE,LO]{\nombredelproyecto}
\fancyfoot[CE,CO]{\leftmark}
\fancyfoot[LE,RO]{\thepage}

\renewcommand{\headrulewidth}{2pt}
\renewcommand{\footrulewidth}{1pt}



\newcommand{\nombredelproyecto}{\textbf{Documento de pruebas}}
\setcounter{secnumdepth}{4}
\makeatletter
\renewcommand{\paragraph}{\@startsection{paragraph}{4}{0ex}%
    {-3.25ex plus -1ex minus -0.2ex}%
    {1.5ex plus 0.2ex}%
    {\normalfont\normalsize\bfseries}}
\makeatother

\title{\nombredelproyecto} 
\author{Jesús Abajo Magro \\
Alejandro Díaz Blázquez \\
Javier Fernández Gamo \\
Andrés Galbán Méndez \\
Alejandro Gómez Molano \\
Jaime Millán Ibáñez Archilla \\
Rodrigo Sosa Saez \\
Alejandro Barrachina Argudo}


\date{\today}

\begin{document}
\begin{titlepage}


    \makeatletter
    \centering
    \vspace*{6\baselineskip}

    %------------------------------------------------
    %	Title
    %------------------------------------------------


    {\huge \underbar{\nombredelproyecto}\par} % Title

    %------------------------------------------------
    %	Subtitle
    %------------------------------------------------

    Ingeniería del software\\
    Ingeniería informática e ingeniería de computadores

    \vspace*{3\baselineskip} % Whitespace under the subtitle


    \begin{figure}[H]
        \centering
        \includegraphics[width=0.5\textwidth]{images/Banqueo.png}
    \end{figure}
    %------------------------------------------------
    %	Editor(s)
    %------------------------------------------------


    \vspace{0.75\baselineskip} % Whitespace before the editors
    \begin{flushright}
        \small\@author\\ % Editor list
    \end{flushright}


    \vspace{0.5\baselineskip} % Whitespace below the editor list


    \vfill % Whitespace between editor names and publisher logo

    \vspace{0.3\baselineskip} % Whitespace under the publisher logo

    \today % Publication year
    \makeatother
\end{titlepage}

\tableofcontents
\newpage

\section*{Introducción}
En este documento se muestran los posibles flujos de funcionamiento de las distintas funciones del programa.

\section{Módulo usuarios}
\subsection{Iniciar sesión}
\begin{table}[H]
    \centering
    \begin{tabular}{|>{\bfseries}l|l|}
        \hline
        Nombre de la función                             & iniciarSesion                                                        \\
        \hline
        Precondición                                     & El usuario, cliente o gestor, debe estar dado de alta en el sistema. \\
        \hline
        Postcondición                                    & Los datos deben ser correctos                                        \\
        \hline
        Flujo principal                                  &
        1.- La función cumple con la precondición.
        2.- Los datos de entrada son el DNI y la contraseña.
        3.- La salida indica que se inicia sesión de forma satisfactoria.
        4.- La función cumple con la postcondición.
        \\
        \hline
        Flujo alternativo 1 \newline (Datos incorrectos) &
        1.- La función cumple con la precondición.
        2.- Los datos de entrada son el DNI y la contraseña.
        3.- La salida muestra en pantalla una ventana de error.
        4.- La función no cumple con la postcondición.                                                                          \\
        \hline
        Flujo alternativo 1 \newline (Usuario no existe) &
        1.- La función no cumple con la precondición.
        2.- Los datos de entrada son el DNI y la contraseña.
        3.- La salida muestra en pantalla una ventana de error.
        4.- La función no cumple con la postcondición.                                                                          \\
    \end{tabular}
\end{table}
\end{document}
