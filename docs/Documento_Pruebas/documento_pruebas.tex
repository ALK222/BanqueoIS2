\documentclass[12pt]{article}
\usepackage[a4paper, margin=1in, headheight=14pt]{geometry}
\usepackage[utf8]{inputenc}
\usepackage[spanish]{babel}
\usepackage{tabularx}
\usepackage{graphicx}
\usepackage{float}
\usepackage{setspace}
\usepackage{anyfontsize}
\usepackage[toc,page]{appendix} % Remove toc,page to render wo appendix
\usepackage{hyperref}
\usepackage[table]{xcolor}
\usepackage{fancyhdr}
\usepackage{nameref}
\usepackage{datetime}
\usepackage{tikz}
\usepackage[11pt]{moresize}
\usepackage{enumerate}
\usepackage{etoolbox}
\usepackage{makecell}

\usetikzlibrary{calc}
\newcommand\HRule{\rule{\textwidth}{1pt}}


\pagestyle{fancy}
\fancyhf{}
\fancyhead[LE,RO]{Grupo 4}
\fancyhead[RE,LO]{\nombredelproyecto}
\fancyfoot[CE,CO]{\leftmark}
\fancyfoot[LE,RO]{\thepage}

\renewcommand{\headrulewidth}{2pt}
\renewcommand{\footrulewidth}{1pt}



\newcommand{\nombredelproyecto}{\textbf{Documento de pruebas}}
\setcounter{secnumdepth}{4}
\makeatletter
\renewcommand{\paragraph}{\@startsection{paragraph}{4}{0ex}%
    {-3.25ex plus -1ex minus -0.2ex}%
    {1.5ex plus 0.2ex}%
    {\normalfont\normalsize\bfseries}}
\makeatother

\title{\nombredelproyecto} 
\author{Jesús Abajo Magro \\
Alejandro Díaz Blázquez \\
Javier Fernández Gamo \\
Andrés Galbán Méndez \\
Alejandro Gómez Molano \\
Jaime Millán Ibáñez Archilla \\
Rodrigo Sosa Saez \\
Alejandro Barrachina Argudo}


\date{\today}

\begin{document}
\begin{titlepage}


    \makeatletter
    \centering
    \vspace*{6\baselineskip}

    %------------------------------------------------
    %	Title
    %------------------------------------------------


    {\huge \underbar{\nombredelproyecto}\par} % Title

    %------------------------------------------------
    %	Subtitle
    %------------------------------------------------

    Ingeniería del software\\
    Ingeniería informática e ingeniería de computadores

    \vspace*{3\baselineskip} % Whitespace under the subtitle


    \begin{figure}[H]
        \centering
        \includegraphics[width=0.5\textwidth]{images/Banqueo.png}
    \end{figure}
    %------------------------------------------------
    %	Editor(s)
    %------------------------------------------------


    \vspace{0.75\baselineskip} % Whitespace before the editors
    \begin{flushright}
        \small\@author\\ % Editor list
    \end{flushright}


    \vspace{0.5\baselineskip} % Whitespace below the editor list


    \vfill % Whitespace between editor names and publisher logo

    \vspace{0.3\baselineskip} % Whitespace under the publisher logo

    \today % Publication year
    \makeatother
\end{titlepage}

\tableofcontents
\newpage

\section*{Introducción}
En este documento se muestran los posibles flujos de funcionamiento de las distintas funciones del programa.

\section{Módulo usuarios}
\subsection{Iniciar sesión}
\begin{table}[H]
    \centering
    \begin{tabularx}{\textwidth}{|>{\bfseries}X|X|}
        \hline
        Nombre de la función                                        & iniciarSesion                                                        \\
        \hline
        Precondición                                                & El usuario, cliente o gestor, debe estar dado de alta en el sistema. \\
        \hline
        Postcondición                                               & Los datos deben ser correctos                                        \\
        \hline
        Flujo principal                                             &
        1.- La función cumple con la precondición.
        2.- Los datos de entrada son el DNI y la contraseña.
        3.- La salida indica que se inicia sesión de forma satisfactoria.
        4.- La función cumple con la postcondición.
        \\
        \hline
        \makecell{Flujo alternativo 1 \newline (Datos incorrectos)} &
        1.- La función cumple con la precondición.\newline
        2.- Los datos de entrada son el DNI y la contraseña.\newline
        3.- La salida muestra en pantalla una ventana de error.\newline
        4.- La función no cumple con la postcondición.\newline                                                                             \\
        \hline
        Flujo alternativo 2 \newline (Usuario no existe)            &
        1.- La función no cumple con la precondición.\newline
        2.- Los datos de entrada son el DNI y la contraseña.\newline
        3.- La salida muestra en pantalla una ventana de error.\newline
        4.- La función no cumple con la postcondición.\newline                                                                             \\
        \hline
    \end{tabularx}
\end{table}
\subsection{Cerrar sesión}
\begin{table}[H]
    \centering
    \begin{tabularx}{\textwidth}{|>{\bfseries}X|X|}
        \hline
        Nombre de la función                             & cerrarSesion                                              \\
        \hline
        Precondición                                     & El usuario, cliente o gestor, debe haber iniciado sesión. \\
        \hline
        Postcondición                                    & Inexistente.                                              \\
        \hline
        Flujo principal                                  &
        1.- La función cumple con la precondición.
        2.- No hay datos de entrada.
        3.- No hay salida.
        4.- No hay postcondición.
        \\
        \hline
        Flujo alternativo 1 \newline (Datos incorrectos) &
        1.- La función no cumple con la precondición.\newline
        2.- No hay datos de entrada.\newline
        3.- La salida muestra en pantalla una ventana de error.\newline
        4.- No hay postcondición.  \newline                                                                          \\
        \hline
        Flujo alternativo 2 \newline (Usuario no existe) &
        1.- La función cumple con la precondición.\newline
        2.- No hay datos de entrada.\newline
        3.- No se hay ninguna salida por la interrupción del programa.\newline
        4.- No hay postcondición.   \newline                                                                         \\
        \hline
    \end{tabularx}
\end{table}

\subsection{Crear usuario}
\begin{table}[H]
    \centering
    \begin{tabularx}{\textwidth}{|>{\bfseries}X|X|}
        \hline
        Nombre de la función                             & altaUsuario                                                                                \\
        \hline
        Precondición                                     & 1.- El gestor inició sesión. \newline 2.- El usuario no debe existir previamente a su alta \\
        \hline
        Postcondición                                    & Los datos del usuario estarán en la base de datos.                                         \\
        \hline
        Flujo principal                                  &
        1.- La función cumple con la precondición.\newline
        2.- Los datos de entrada son los datos del nuevo usuario.\newline
        3.- La salida muestra una pantalla diciendo que el usuario se ha creado satisfactoriamente.\newline
        4.- La función cumple con la postcondición.\newline
        \\
        \hline
        Flujo alternativo 1 \newline (Datos incorrectos) &
        1.- La función cumple no con la precondición.\newline
        2.- Los datos de entrada son los datos del nuevo usuario.\newline
        3.- La salida muestra una pantalla diciendo que el usuario ya existe.\newline
        4.- La función cumple con la postcondición.\newline                                                                                           \\
        \hline
    \end{tabularx}
\end{table}

\subsection{Eliminar usuario}
\begin{table}[H]
    \centering
    \begin{tabularx}{\textwidth}{|>{\bfseries}X|X|}
        \hline
        Nombre de la función                             & bajaUsuario                                                                             \\
        \hline
        Precondición                                     & 1.- El gestor inició sesión. \newline 2.- El usuario debe existir previamente a su alta \\
        \hline
        Postcondición                                    & Los datos del usuario no estarán en la base de datos.                                   \\
        \hline
        Flujo principal                                  &
        1.- La función cumple con la precondición.\newline
        2.- Los datos de entrada son los datos del usuario a dar de baja.\newline
        3.- La salida muestra una pantalla diciendo que el usuario se ha eliminado satisfactoriamente.\newline
        4.- La función cumple con la postcondición.\newline
        \\
        \hline
        Flujo alternativo 1 \newline (Usuario no existe) &
        1.- La función cumple no con la precondición.\newline
        2.- Los datos de entrada son los datos del usuario a dar de baja.\newline
        3.- La salida muestra una pantalla diciendo que el usuario no existe.\newline
        4.- La función cumple con la postcondición.\newline                                                                                        \\
        \hline
    \end{tabularx}
\end{table}

\subsection{Consultar la lista de usuarios}
\begin{table}[H]
    \centering
    \begin{tabularx}{\textwidth}{|>{\bfseries}X|X|}
        \hline
        Nombre de la función                                              & consultarListaUsuarios                                                                    \\
        \hline
        Precondición                                                      & 1.- El gestor inició sesión. \newline 2.- Los parámetros de búsqueda deben ser correctos. \\
        \hline
        Postcondición                                                     & Se muestran los usuarios que pasan por el filtro.                                         \\
        \hline
        Flujo principal                                                   &
        1.- La función cumple con la precondición.\newline
        2.- Los datos de entrada son dos string: uno con la “clave” y el otro con el modo de filtraje.\newline
        3.- La salida muestra una lista con los usuarios que pasan el filtro.\newline
        4.- La función cumple la postcondición.\newline
        \\
        \hline
        Flujo alternativo 1 \newline (Parámetros de búsqueda incorrectos) &
        1.- La función cumple no con la precondición.\newline
        2.- Los datos de entrada son un string “clave” y el modo de filtraje.\newline
        3.- La salida muestra una pantalla de error.\newline
        4.- La función no cumple la postcondición.\newline                                                                                                            \\
        \hline
        Flujo alternativo 2 \newline (Ningún usuario pasa el filtro)      &
        1.- La función cumple no con la precondición.\newline
        2.- Los datos de entrada son un string “clave” y el modo de filtraje.\newline
        3.- La salida muestra una lista vacía, ya que ningún usuario pasa el filtro.\newline
        4.- La función no cumple la postcondición.\newline                                                                                                            \\
        \hline
    \end{tabularx}
\end{table}

\subsection{Buscar usuario}

\begin{table}[H]
    \centering
    \begin{tabularx}{\textwidth}{|>{\bfseries}X|X|}
        \hline
        Nombre de la función                                                       & buscarUsuario                                                                            \\
        \hline
        Precondición                                                               & 1.- El gestor inició sesión. \newline 2.- El usuario debe existir previamente a su alta. \\
        \hline
        Postcondición                                                              & Se mostrarán los datos de dicho usuario.                                                 \\
        \hline
        Flujo principal                                                            &
        1.- La función cumple con la precondición.\newline
        2.- El dato de entrada es el DNI del usuario a buscar.\newline
        3.- La salida muestra los datos del usuario.\newline
        4.- La función cumple con la postcondición.\newline
        \\
        \hline
        Flujo alternativo 1 \newline (El DNI no se corresponde con ningún usuario) &
        1.- La función cumple no con la precondición.\newline
        2.- Los datos de entrada son los datos del usuario a buscar.\newline
        3.- La salida muestra una pantalla diciendo que el usuario no existe.\newline
        4.- La función no cumple con la postcondición.\newline                                                                                                                \\
        \hline
    \end{tabularx}
\end{table}

\subsection{Modificar Usuarios}
\begin{table}[H]
    \centering
    \begin{tabularx}{\textwidth}{|>{\bfseries}X|X|}
        \hline
        Nombre de la función                                     & modificarUsuario                                                                         \\
        \hline
        Precondición                                             & 1.- El gestor inició sesión. \newline 2.- El usuario debe existir previamente a su alta. \\
        \hline
        Postcondición                                            & Se modifican los datos del usuario en la base de datos.                                  \\
        \hline
        Flujo principal                                          &
        1.- La función cumple con la precondición.\newline
        2.- El dato de entrada es el usuario modificado.\newline
        3.- La salida muestra una ventana que indica que la operación se ha realizado correctamente.\newline
        4.- La función cumple con la postcondición.\newline
        \\
        \hline
        Flujo alternativo 1 \newline (No se guardan los cambios) &
        1.- La función cumple no con la precondición.\newline
        2.- El dato de entrada es el usuario modificado.\newline
        3.- La salida muestra una pantalla de error.\newline
        4.- La función no cumple con la postcondición.\newline                                                                                              \\
        \hline
    \end{tabularx}
\end{table}

\section{Módulo cuentas}
\subsection{Crear cuenta}
\begin{table}[H]
    \centering
    \begin{tabularx}{\textwidth}{|>{\bfseries}X|X|}
        \hline
        Nombre de la función                               & altaCuenta                                                                                \\
        \hline
        Precondición                                       & 1.- El gestor inició sesión. \newline 2.- La cuenta no debe existir previamente a su alta \\
        \hline
        Postcondición                                      & Los datos de la cuenta estarán en la base de datos.                                       \\
        \hline
        Flujo principal                                    &
        1.- La función cumple con la precondición.\newline
        2.- Los datos de entrada son los datos de la nueva cuenta.\newline
        3.- La salida muestra una pantalla diciendo que la cuenta se ha creado satisfactoriamente.\newline
        4.- La función cumple con la postcondición.\newline
        \\
        \hline
        Flujo alternativo 1 \newline (La cuenta ya existe) &
        1.- La función cumple no con la precondición.\newline
        2.- Los datos de entrada son los datos de la nuevo cuenta.\newline
        3.- La salida muestra una pantalla diciendo que la cuenta ya existe.\newline
        4.- La función cumple con la postcondición.\newline                                                                                            \\
        \hline
    \end{tabularx}
\end{table}

\subsection{Eliminar cuenta}
\begin{table}[H]
    \centering
    \begin{tabularx}{\textwidth}{|>{\bfseries}X|X|}
        \hline
        Nombre de la función                            & bajaCuenta                                                                              \\
        \hline
        Precondición                                    & 1.- El gestor inició sesión. \newline 2.- La cuenta debe existir previamente a su alta. \\
        \hline
        Postcondición                                   & Los datos del usuario no estarán en la base de datos.                                   \\
        \hline
        Flujo principal                                 &
        1.- La función cumple con la precondición.\newline
        2.- Los datos de entrada son los datos de la cuenta a dar de baja.\newline
        3.- La salida muestra una pantalla diciendo que la cuenta se ha eliminado satisfactoriamente.\newline
        4.- La función cumple con la postcondición.\newline
        \\
        \hline
        Flujo alternativo 1 \newline (Cuenta no existe) &
        1.- La función cumple no con la precondición.\newline
        2.- Los datos de entrada son los datos de la cuenta a dar de baja.\newline
        3.- La salida muestra una pantalla diciendo que la cuenta no existe.\newline
        4.- La función cumple con la postcondición.\newline                                                                                       \\
        \hline
    \end{tabularx}
\end{table}

\subsection{Consultar la lista de cuentas}
\begin{table}[H]
    \centering
    \begin{tabularx}{\textwidth}{|>{\bfseries}X|X|}
        \hline
        Nombre de la función                                              & consultarCuentas                                                                           \\
        \hline
        Precondición                                                      & 1.- El usuario inició sesión. \newline 2.- Los parámetros de búsqueda deben ser correctos. \\
        \hline
        Postcondición                                                     & Se muestran las cuentas que pasan por el filtro.                                           \\
        \hline
        Flujo principal                                                   &
        1.- La función cumple con la precondición.\newline
        2.- Los datos de entrada son dos string: uno con la DNI del titular y el otro su nombre.\newline
        3.- La salida muestra una lista con las cuentas que pasan el filtro.\newline
        4.- La función cumple la postcondición.\newline
        \\
        \hline
        Flujo alternativo 1 \newline (Parámetros de búsqueda incorrectos) &
        1.- La función cumple no con la precondición.\newline
        2.- Los datos de entrada son un string DNI y el nombre.\newline
        3.- La salida muestra una pantalla de error.\newline
        4.- La función no cumple la postcondición.\newline                                                                                                             \\
        \hline
        Flujo alternativo 2 \newline (Ningún usuario pasa el filtro)      &
        1.- La función cumple no con la precondición.\newline
        2.- Los datos de entrada son dos string: uno con la DNI del titular y el otro su nombre.\newline
        3.- La salida muestra una lista vacía, ya que ninguna cuenta pasa el filtro.\newline
        4.- La función no cumple la postcondición.\newline                                                                                                             \\
        \hline
    \end{tabularx}
\end{table}

\subsection{Buscar cuenta}

\begin{table}[H]
    \centering
    \begin{tabularx}{\textwidth}{|>{\bfseries}X|X|}
        \hline
        Nombre de la función                                                                        & buscarCuenta                                                                            \\
        \hline
        Precondición                                                                                & 1.- El gestor inició sesión. \newline 2.- La cuenta debe existir previamente a su alta. \\
        \hline
        Postcondición                                                                               & Se mostrarán los datos de dicha cuenta.                                                 \\
        \hline
        Flujo principal                                                                             &
        1.- La función cumple con la precondición.\newline
        2.- El dato de entrada es el número de referencia de la cuenta a buscar.\newline
        3.- La salida muestra los datos de la cuenta.\newline
        4.- La función cumple con la postcondición.\newline
        \\
        \hline
        Flujo alternativo 1 \newline (El número de referencia no se corresponde con ninguna cuenta) &
        1.- La función cumple no con la precondición.\newline
        2.- El dato de entrada es el número de referencia de la cuenta a buscar.\newline
        3.- La salida muestra una pantalla diciendo que la cuenta no existe.\newline
        4.- La función no cumple con la postcondición.\newline                                                                                                                                \\
        \hline
    \end{tabularx}
\end{table}

\subsection{Modificar Usuarios}
\begin{table}[H]
    \centering
    \begin{tabularx}{\textwidth}{|>{\bfseries}X|X|}
        \hline
        Nombre de la función                                     & modificarCuenta                                                                         \\
        \hline
        Precondición                                             & 1.- El gestor inició sesión. \newline 2.- La cuenta debe existir previamente a su alta. \\
        \hline
        Postcondición                                            & Se modifican los datos de la cuenta en la base de datos.                                \\
        \hline
        Flujo principal                                          &
        1.- La función cumple con la precondición.\newline
        2.- El dato de entrada es la cuenta modificada.\newline
        3.- La salida muestra una ventana que indica que la operación se ha realizado correctamente.\newline
        4.- La función cumple con la postcondición.\newline
        \\
        \hline
        Flujo alternativo 1 \newline (No se guardan los cambios) &
        1.- La función cumple no con la precondición.\newline
        2.- El dato de entrada es la cuenta modificada.\newline
        3.- La salida muestra una pantalla de error.\newline
        4.- La función no cumple con la postcondición.\newline                                                                                             \\
        \hline
    \end{tabularx}
\end{table}

\subsection{Consultar movimientos}
\begin{table}[H]
    \centering
    \begin{tabularx}{\textwidth}{|>{\bfseries}X|X|}
        \hline
        Nombre de la función                                            & getMovimientos                                                                         \\
        \hline
        Precondición                                                    & 1.- El usuario inició sesión. \newline 2.- La cuenta debe existir en la base de datos. \\
        \hline
        Postcondición                                                   & Se muestran los movimientos que pasan por el filtro.                                   \\
        \hline
        Flujo principal                                                 &
        1.- La función cumple con la precondición.\newline
        2.- Los datos de entrada son los meses a consultar y la cuenta.\newline
        3.- La salida muestra una lista con las cuentas que pasan el filtro.\newline
        4.- La función cumple la postcondición.\newline
        \\
        \hline
        Flujo alternativo 1 \newline (La cuenta es errónea)             &
        1.- La función cumple no con la precondición.\newline
        2.- Los datos de entrada son los meses a consultar y la cuenta.\newline
        3.- La salida muestra una pantalla de error.\newline
        4.- La función no cumple la postcondición.\newline                                                                                                       \\
        \hline
        Flujo alternativo 2 \newline (Ningún movimiento pasa el filtro) &
        1.- La función cumple no con la precondición.\newline
        2.- Los datos de entrada son los meses a consultar y la cuenta.\newline
        3.- La salida muestra una lista vacía, ya que ningún movimiento pasa el filtro.\newline
        4.- La función no cumple la postcondición.\newline                                                                                                       \\
        \hline
    \end{tabularx}
\end{table}
\end{document}
