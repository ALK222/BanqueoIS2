\documentclass[12pt]{article}
\usepackage[a4paper, margin=1in, headheight=14pt]{geometry}
\usepackage[utf8]{inputenc}
\usepackage[spanish]{babel}
\usepackage{tabularx}
\usepackage{graphicx}
\usepackage{float}
\usepackage{setspace}
\usepackage{anyfontsize}
\usepackage[toc,page]{appendix} % Remove toc,page to render wo appendix
\usepackage{hyperref}
\usepackage[table]{xcolor}
\usepackage{fancyhdr}
\usepackage{nameref}
\usepackage{datetime}
\usepackage{tikz}
\usepackage[11pt]{moresize}
\usepackage{enumerate}
\usepackage{etoolbox}


\usetikzlibrary{calc}
\newcommand\HRule{\rule{\textwidth}{1pt}}


\pagestyle{fancy}
\fancyhf{}
\fancyhead[LE,RO]{Grupo 4}
\fancyhead[RE,LO]{\nombredelproyecto}
\fancyfoot[CE,CO]{\leftmark}
\fancyfoot[LE,RO]{\thepage}

\renewcommand{\headrulewidth}{2pt}
\renewcommand{\footrulewidth}{1pt}



\newcommand{\nombredelproyecto}{\textbf{Memoria}}
\setcounter{secnumdepth}{4}
\makeatletter
\renewcommand{\paragraph}{\@startsection{paragraph}{4}{0ex}%
    {-3.25ex plus -1ex minus -0.2ex}%
    {1.5ex plus 0.2ex}%
    {\normalfont\normalsize\bfseries}}
\makeatother

\title{\nombredelproyecto} 
\author{Jesús Abajo Magro \\
Alejandro Díaz Blázquez \\
Javier Fernández Gamo \\
Andrés Galbán Méndez \\
Alejandro Gómez Molano \\
Jaime Millán Ibáñez Archilla \\
Rodrigo Sosa Saez \\
Alejandro Barrachina Argudo}


\date{\today}

\begin{document}
\begin{titlepage}


    \makeatletter
    \centering
    \vspace*{6\baselineskip}

    %------------------------------------------------
    %	Title
    %------------------------------------------------


    {\huge \underbar{\nombredelproyecto}\par} % Title

    %------------------------------------------------
    %	Subtitle
    %------------------------------------------------

    Ingeniería del software\\
    Ingeniería informática e ingeniería de computadores

    \vspace*{3\baselineskip} % Whitespace under the subtitle


    \begin{figure}[H]
        \centering
        \includegraphics[width=0.5\textwidth]{images/Banqueo.png}
    \end{figure}
    %------------------------------------------------
    %	Editor(s)
    %------------------------------------------------


    \vspace{0.75\baselineskip} % Whitespace before the editors
    \begin{flushright}
        \small\@author\\ % Editor list
    \end{flushright}


    \vspace{0.5\baselineskip} % Whitespace below the editor list


    \vfill % Whitespace between editor names and publisher logo

    \vspace{0.3\baselineskip} % Whitespace under the publisher logo

    \today % Publication year
    \makeatother
\end{titlepage}

\tableofcontents
\newpage
\section{Introducción}
El objetivo de este proyecto era la creación de una aplicación informática que permitiera la administración de un banco (``Banqueo''), de la forma más realista posible.

En este documento se explica brevemente cómo están estructurados los archivos entregados y lo que se encuentra en cada uno de ellos. Todo el proyecto está comprimido en un fichero .zip que contiene 5 carpetas, correspondientes a los apartados del 2 al 6 de esta memoria, donde se explica con más detalle qué contienen.


\section{Diagramas de clases, componentes y secuencia}

En esta carpeta se encuentran dos archivos .pdf que recogen todos los diagramas usados para hacer la aplicación.
\subsection{Diagramas de clases y componentes}

En este documento se muestran el diagrama de componentes y los diagramas de clases de cada parte del programa, siendo este el modelo que se ha seguido para realizar la aplicación.
\subsection{Diagramas de secuencia y de clases GUI}

En este documento se muestran los diagramas de secuencia y los diagramas de clase de las interfaces gráficas de la aplicación.


\section{Código de la aplicación}
En esta carpeta se encuentran todos los archivos necesarios para ejecutar la aplicación, incluido un fichero .jar ejecutable.

\section{Aplicación lista para ejecutar}
En esta carpeta se encuentra el .jar del programa junto a la carpeta \textit{resources} y un archivo \textit{README} con instrucciones.

\section{Documentación del código}
En esta carpeta se encuentra la documentación del código, generada con Javadoc.


\section{Documento de pruebas}
En esta carpeta se encuentra el documento de pruebas de la aplicación, en formato .pdf.


\section{Memoria}
En esta carpeta se encuentra la memoria de la aplicación``Banqueo'' en formato .pdf, es decir, este documento.


\section*{Comentarios sobre cómo mejorar la asignatura}
La parte práctica de la asignatura, como realizar este proyecto, resulta bastante entretenida y una buena forma de aplicar y practicar el contenido que se ve en clase. Las varias entregas previas de los documentos, que se corrijan y se expliquen los errores para poder corregirlos también es un acierto.

La parte teórica puede resultar pesada, pero es la única forma de darla, y como se ha dicho anteriormente, con el proyecto se hace más amena.
\end{document}
